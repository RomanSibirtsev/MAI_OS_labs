\documentclass[a4paper, 12pt]{article}
\usepackage{cmap}
\usepackage[12pt]{extsizes}			
\usepackage{mathtext} 				
\usepackage[T2A]{fontenc}			
\usepackage[utf8]{inputenc}			
\usepackage[english,russian]{babel}
\usepackage{setspace}
\singlespacing
\usepackage{amsmath,amsfonts,amssymb,amsthm,mathtools}
\usepackage{fancyhdr}
\usepackage{soulutf8}
\usepackage{euscript}
\usepackage{mathrsfs}
\usepackage{listings}

\usepackage[colorlinks=true, urlcolor=blue, linkcolor=black]{hyperref}

\pagestyle{fancy}
\usepackage{indentfirst}
\usepackage[top=10mm]{geometry}
\rhead{}
\lhead{}
\renewcommand{\headrulewidth}{0mm}
\usepackage{tocloft}
\renewcommand{\cftsecleader}{\cftdotfill{\cftdotsep}}
\usepackage[dvipsnames]{xcolor}

\lstdefinestyle{mystyle}{ 
	keywordstyle=\color{OliveGreen},
	numberstyle=\tiny\color{Gray},
	stringstyle=\color{BurntOrange},
	basicstyle=\ttfamily\footnotesize,
	breakatwhitespace=false,         
	breaklines=true,                 
	captionpos=b,                    
	keepspaces=true,                 
	numbers=left,                    
	numbersep=5pt,                  
	showspaces=false,                
	showstringspaces=false,
	showtabs=false,                  
	tabsize=2
}

\lstset{style=mystyle}

\begin{document}
\thispagestyle{empty}	
\begin{center}
	Московский авиационный институт
	
	(Национальный исследовательский университет)
	
	Факультет "Информационные технологии и прикладная математика"
	
	Кафедра "Вычислительная математика и программирование"
	
\end{center}
\vspace{40ex}
\begin{center}
	\textbf{\large{Лабораторная работа №4 по курсу\linebreak \textquotedblleft Операционные системы\textquotedblright}}
\end{center}
\vspace{35ex}
\begin{flushright}
	\textit{Студент: } Сибирцев Роман Денисович
	
	\vspace{2ex}
	\textit{Группа: } М8О-208Б-22
	
	\vspace{2ex}
	\textit{Преподаватель: } Миронов Евгений Сергеевич
	
	\vspace{2ex}
	\textit{Вариант: } 16 
	
	\vspace{2ex}
	\textit{Оценка: } \underline{\quad\quad\quad\quad\quad\quad}
	
	 \vspace{2ex}
	\textit{Дата: } \underline{\quad\quad\quad\quad\quad\quad}
	
	\vspace{2ex}
	\textit{Подпись: } \underline{\quad\quad\quad\quad\quad\quad}
	
\end{flushright}

\vspace{5ex}

\begin{vfill}
	\begin{center}
		Москва, 2023
	\end{center}	
\end{vfill}
\newpage

\begingroup
\color{black}
\tableofcontents\newpage
\endgroup

\section{Репозиторий}
\href{https://github.com/RomanSibirtsev/MAI_OS_labs}{https://github.com/RomanSibirtsev/MAI\_OS\_labs}

\section{Цель работы}
Приобретение практических навыков в:
\begin{itemize}
  \item Создании динамических библиотек
  \item Создании программ, которые используют функции динамических библиотек
\end{itemize}

\section{Задание}
Требуется создать динамические библиотеки, которые реализуют определенный функционал. Далее использовать данные библиотеки 2-мя способами:
\begin{enumerate}
  \item Во время компиляции (на этапе «линковки»/linking)
  \item Во время исполнения программы. Библиотеки загружаются в память с помощью интерфейса ОС для работы с динамическими библиотеками
\end{enumerate}
В конечном итоге, в лабораторной работе необходимо получить следующие части:
\begin{itemize}
  \item Динамические библиотеки, реализующие контракты, которые заданы вариантом;
  \item Тестовая программа (программа №1), которая используют одну из библиотек, используя знания полученные на этапе компиляции;
  \item Тестовая программа (программа №2), которая загружает библиотеки, используя только их местоположение и контракты.
\end{itemize}

\section{Описание работы программы}
Функции, написанные в результате выполнения лабораторной работы:
\begin{itemize}
  \item Подсчёт количества простых чисел на отрезке [A, B] (A, B - натуральные)
  \item Подсчёт наибольшего общего делителя для двух натуральных чисел
\end{itemize}

В ходе выполнения лабораторной работы я использовала следующие системные вызовы:
\begin{itemize}
  \item dlopen - открытие динамического объекта
  \item dlclose - закрытие динамического объекта
\end{itemize}


\newpage

\section{Исходный код}
lib.hpp
\begin{lstlisting}[language=C++]
#pragma once 

#ifdef __cplusplus
extern "C" {
#endif

int PrimeCount(int a, int b);
int GCF(int a, int b);

#ifdef __cplusplus
}
#endif
\end{lstlisting}

lib1.cpp
\begin{lstlisting}[language=C++]
#include "lib.hpp"
#include <iostream>
extern "C" bool IsPrime(int a) {
    bool res = 1;
    for (int i = 2; i < a; ++i) {
        if (a % i == 0) {
            res = 0;
        }
    }
    return res;
} 

extern "C" int PrimeCount(int a, int b) {
    int count = 0;
    for (int i = a; i <= b; ++i) {
        if (IsPrime(i)) {
            ++count;
        }
    }
    return count;
}
extern "C" int GCF(int a, int b) {
    while (a != 0 && b != 0) {
        if (a > b) {
            a = a % b;
        } else {
            b = b % a;
        }
    }
    return a + b;
}
\end{lstlisting}

lib2.cpp
\begin{lstlisting}[language=C++]
#include "lib.hpp"
#include <iostream>
#include <vector>

extern "C" int PrimeCount(int a, int b) {
    std::vector<bool> prime(b + 1, true);
    prime[0] = prime[1] = false;
    for (int i = 2; i * i <= b + 1; ++i) {
        for (int j = i * i; j <= b + 1; j += i) {
            prime[j] = false;
        }
    }
    int count = 0;
    for (int i = a; i <= b; ++i) {
        if (prime[i]) {
            ++count;
        }
    }
    return count;
}
extern "C" int GCF(int a, int b) {
    int res = 0;
    for (int i = std::min(a, b); i > 0; --i) {
        if (a % i == 0 && b % i == 0) {
            res = i;
        }
    }
    return res;
}
\end{lstlisting}

main\_dynamic.cpp
\begin{lstlisting}[language=C++]
#include <iostream>
#include <cstdlib>
#include <dlfcn.h>

using PrimeCountFunc = int (*)(int, int);
using GCFFunc = int (*)(int, int);

void* loadLibrary(const std::string& libraryName) {
    void* handle = dlopen(libraryName.c_str(), RTLD_LAZY);
    if (!handle) {
        std::cerr << "Cannot load library: " << dlerror() << std::endl;
        exit(EXIT_FAILURE);
    }
    return handle;
}

void unloadLibrary(void* handle) {
    if (dlclose(handle) != 0) {
        std::cerr << "Cannot unload library: " << dlerror() << std::endl;
        exit(EXIT_FAILURE);
    }
}

int main() {
    int whichLib = 1;

    const char* PathToLib1 = getenv("PATH_TO_LIB1");
    const char* PathToLib2 = getenv("PATH_TO_LIB2");

    if (PathToLib1 == nullptr || PathToLib2 == nullptr) {
        std::cout << "PATH_TO_LIB1 or PATH_TO_LIB2 is not specified\n";
        exit(1);
    }

    void* libraryHandle = loadLibrary(PathToLib1);

    PrimeCountFunc PrimeCount = (PrimeCountFunc)dlsym(libraryHandle, "PrimeCount");
    GCFFunc GCF = (GCFFunc)dlsym(libraryHandle, "GCF");
    std::string command;

    while (true) {
        std::cin >> command;
        int a, b;
        if (command == "0") {
            unloadLibrary(libraryHandle);
            if (whichLib == 2) {
                libraryHandle = loadLibrary(PathToLib1);
                whichLib = 1;
            } else if (whichLib == 1) {
                libraryHandle = loadLibrary(PathToLib2);
                whichLib = 2;
            }
            PrimeCount = reinterpret_cast<PrimeCountFunc>(dlsym(libraryHandle, "PrimeCount"));
            GCF = reinterpret_cast<GCFFunc>(dlsym(libraryHandle, "GCF"));

        } else if (command == "1") {
            std::cin >> a >> b;
            std::cout << "OK\n";
            std::cout << "PrimeCount = " << PrimeCount(a, b) << std::endl;
        } else if (command == "2") {
            std::cin >> a >> b;
            std::cout << "OK\n";
            std::cout << "GCF = " << GCF(a, b) << std::endl;
        } else {
            std::cout << "Wrong Argument" << std::endl;
        }
    }

}
\end{lstlisting}

main\_static.cpp
\begin{lstlisting}[language=C++]
#include <iostream>
#include "lib.hpp"

int main() {
    std::string command;
    int a, b;

    while (true) {
        std::cin >> command;
        if (command == "0") {
            break;
        } else if (command == "1") {
            std::cin >> a >> b;
            std::cout << "PrimeCount = " << PrimeCount(a, b) << std::endl;
        } else if (command == "2") {
            std::cin >> a >> b;
            std::cout << "GCF = " << GCF(a, b) << std::endl;
        } else {
            std::cout << "Wrong Argument" << std::endl;
        }
    }
}
\end{lstlisting}

\newpage
\section{Тесты}
\begin{lstlisting}[language=C++]

\end{lstlisting}

\begin{lstlisting}[language=C++]

\end{lstlisting}
\newpage

\section{Запуск тестов}
\begin{verbatim}

\end{verbatim}
\newpage

\section{Выводы}

В результате выполнения данной лабораторной работы были созданы динамические библиотеки, которые реализуют функционал в соответствие с вариантом задания на С++. Я приобрел практические навыки в создании программ, которые используют функции динамических библиотек.
\end{document}